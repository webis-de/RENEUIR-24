\section{Related Work}

Why have we choosen model X and model Y for our work?

From~\cite{jiao20}: Language model pre-training, such as BERT, has significantly improved the performances of many natural language processing tasks. However, pre-trained language models are usually computationally expensive, so it is difficult to efficiently execute them on resource-restricted devices. To accelerate inference and reduce model size while maintaining accuracy, we first propose a novel Transformer distillation method that is specially designed for knowledge distillation (KD) of the Transformer-based models. By leveraging this new KD method, the plenty of knowledge encoded in a large teacher BERT can be effectively transferred to a small student Tiny-BERT. Then, we introduce a new two-stage learning framework for TinyBERT, which performs Transformer distillation at both the pretraining and task-specific learning stages. This framework ensures that TinyBERT can capture he general-domain as well as the task-specific knowledge in BERT.
TinyBERT with 4 layers is empirically effective and achieves more than 96.8\% the performance of its teacher BERTBASE on GLUE benchmark, while being 7.5x smaller and 9.4x faster on inference. TinyBERT with 4 layers is also significantly better than 4-layer state-of-the-art baselines on BERT distillation, with only about 28\% parameters and about 31\% inference time of them. Moreover, TinyBERT with 6 layers performs on-par with its teacher BERTBASE.


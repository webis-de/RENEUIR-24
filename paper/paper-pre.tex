\title{Team FSU at ReNeuIR 2024}
\subtitle{Document Re-Ranking with Tiny Cross Encoders}
%\subtitle{Does it work with TinyBERT? Testing X,y and Z for xxyyx.}


%%
%% Rights management information.
%% CC-BY is default license.
\copyrightyear{2024}
\copyrightclause{Copyright for this paper by its authors.
  Use permitted under Creative Commons License Attribution 4.0
  International (CC BY 4.0).}

%%
%% This command is for the conference information
\conference{Woodstock'22: Symposium on the irreproducible science,
  June 07--11, 2022, Woodstock, NY}


\tnotemark[1]
\tnotetext[1]{You can use this document as the template for preparing your
  publication. We recommend using the latest version of the ceurart style.}

%%
%% The "author" command and its associated commands are used to define
%% the authors and their affiliations.
\author[1,2]{Dmitry S. Kulyabov}[%
orcid=0000-0002-0877-7063,
email=kulyabov-ds@rudn.ru,
url=https://yamadharma.github.io/,
]
\cormark[1]
\fnmark[1]
\address[1]{Peoples' Friendship University of Russia (RUDN University),
  6 Miklukho-Maklaya St, Moscow, 117198, Russian Federation}
\address[2]{Joint Institute for Nuclear Research,
  6 Joliot-Curie, Dubna, Moscow region, 141980, Russian Federation}

\author[3]{Ilaria Tiddi}[%
orcid=0000-0001-7116-9338,
email=i.tiddi@vu.nl,
url=https://kmitd.github.io/ilaria/,
]
\fnmark[1]
\address[3]{Vrije Universiteit Amsterdam, De Boelelaan 1105, 1081 HV Amsterdam, The Netherlands}

\author[4]{Manfred Jeusfeld}[%
orcid=0000-0002-9421-8566,
email=Manfred.Jeusfeld@acm.org,
url=http://conceptbase.sourceforge.net/mjf/,
]
\fnmark[1]
\address[4]{University of Skövde, Högskolevägen 1, 541 28 Skövde, Sweden}

%% Footnotes
\cortext[1]{Corresponding author.}
\fntext[1]{These authors contributed equally.}

%%
%% The abstract is a short summary of the work to be presented in the
%% article.
\begin{abstract}
% General Statement
% - distillation +/- score injection improve cross-encoder
%
% Specific Problem
% - unclear, if the two approaches transfer to tiny/smaller models
% Outline proposed solution
% Transfer two approaches to TinyBERT
%
% Evaluation: clear
%
  We describe the submission of team FSU in the ReNeuIR~2024 shared task on reaching efficiency in neural information retrieval. Previous work has shown that distillation and score injection are two approaches to make cross-encoders more effective. Still, the cross-encoders used in this previous works were rather big and hence not the most ``efficient'' ones. try to transfer two techniques that help to improve the effectiveness of cross-encoders to tiny models. We transfer {\color{red} ...}. Our experiments show {\color{red}...}.
%
\end{abstract}

%%
%% Keywords. The author(s) should pick words that accurately describe
%% the work being presented. Separate the keywords with commas.
\begin{keywords}
  Tiny Cross Encoder \sep
  Efficient Re-Ranking \sep
  Distillation \sep
  BM25 Score Injection
\end{keywords}

%%
%% This command processes the author and affiliation and title
%% information and builds the first part of the formatted document.
\maketitle 
